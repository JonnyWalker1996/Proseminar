%% content.tex
%%

%% ==============================
\chapter{Abstract}
\label{ch:Abstract}
%% ==============================

With the ever increasing amount of data to be rendered for a single given scene, experimenting with new approaches for handling and structuring that data becomes more and more important. This paper covers the approach to enhance response times and quality using out-of-core techniques. \\

With the help of out-of-core techniques, it is possible to achieve real-time rendering performance on current-generation graphics cards by overcoming limits posed by the graphics cards' internal memory \cite{Crassin:2009:GRS:1507149.1507152}. Thus, this topic is essential for everyone hoping to achieve such performance for scenes that cannot be efficiently rendered using conventional methods. \\

I will discuss different strategies used to load necessary data chunks into working memory taking into account its finiteness using intelligent data streaming. These include visibility culling and virtual texturing. Also, this paper will give an overview of the possibility to use volumetric texture units called ``Voxels'' instead of regular texels and the advantages and challenges that come with it. Lastly, the topic will be further explored using the examples \cite{Crassin:2009:GRS:1507149.1507152} and \cite{van2009id}.
%



\chapter{Motivation}
\label{ch:Motivation}

Here I will give examples for problems solved by out of core rendering algorithms. I will extract the specific technical limitations of not using the technique from the examples.

\chapter{Basics}
\label{ch:Basics}

Here I will introduce key terminology used. I will explain the terms ``Texture Mapping'', ``Memory Virtualization'' and ``Virtual Texture''.
%% ==============

\chapter{Main Content}
\label{ch:MainContent}

\section{Memory Management}
%% ==============
Here I will talk about ways to utilize slow external memory to be able to store large datasets without losing wanted performance. I will refer to voxels, virtual paging and other techniques.
%% content.tex
%%

%% ==============
\section{Voxels}

Here I will give a detailed overview of the concept of voxels. I will talk about their differences to texels, advantages, challenges and their suitability for out of core processes.

\section{Example 1: GigaVoxels}
%% ==============
Here I will use the discussed concepts to explain how GigaVoxels can be used to render large volumetric datasets with high performance. I will explain how this is related to both voxels and out of core processes.

\section{Example 2: ID Tech 5 Challenges}
Here I will discuss in detail how out of core rendering is compatible with parallelization using the ID Tech 5 Challenges example. I will talk about the faced challenges and how they were solved. 

\section{Example 3: Visualization of very large Landscapes}
Here I will summarize the findings from \cite{10.1007/978-3-540-40014-1_3}. I will refer to multi-resolution algorithms and how they are related to the topic.

\chapter{Epilogue}
\label{ch:Epilogue}

Here I will outline how out of core techniques can be used to develop new ways of data provision and rendering scenes in the future. 